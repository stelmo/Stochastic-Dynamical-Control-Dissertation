\documentclass[11pt,fleqn]{article}
\linespread{1.3}
\usepackage{parskip}
\setlength{\parindent}{0pt} %no paragraph indentation
\setlength{\parskip}{2.1ex plus 0.2ex minus 0.2ex} %3x paragraph spacing
\usepackage{geometry}
\geometry{a4paper,left=30mm,right=25mm,top=20mm,bottom=20mm} %margin spacing
\usepackage{fancyhdr}
\pagestyle{fancy}
\fancyhf{}
\renewcommand{\headrulewidth}{0pt}
\rfoot{\thepage}

\usepackage[pdftex]{graphicx} %so that eps files may be included

\usepackage{amsmath}
\usepackage{amssymb}
\usepackage{amsthm}
\usepackage{float}

\title{Research Plan}
\author{St. Elmo Wilken}
\setcounter{section}{0}

\begin{document}
\maketitle

\section{Overview}
The end goal of my masters is to implement an MPC controller which takes into account uncertainty in system conditions when evaluating the optimal control strategy. I will assume that I have a graphical model structure representing my system and the required prior and conditional distributions. Thus, the main focus will be applying the inference algorithms and combining them with a controller.

To achieve this goal I will spend most of my time working on inference strategies of dynamical systems. I will follow the book by Barber (and to a lesser extent the work by Murphy) closely. Only near the end will I incorporate the controller aspects. I'll be writing up as I proceed.

\subsection{February}
Get large parts of the background written up.
\begin{figure}[H] 
\centering
\includegraphics[scale=1.0]{./imgs/hmm_diagram}
\label{fig_hmm}
\end{figure}
\begin{enumerate}
\item
Hidden Markov Model - discrete: filtering, smoothing and prediction on a toy problem.
\item 
Kalman Filter Model - continuous: filtering, smoothing and prediction on a toy problem.
\end{enumerate}
\pagebreak

\subsection{March}
Complex PGMs - modelling + inference.
\begin{figure}[H] 
\centering
\includegraphics[scale=1.0]{./imgs/switch}
\caption{Switching Kalman Filter}
\label{fig_switch}
\end{figure}
The aim of this section is to study PF and GSF methods on a switching Kalman model. These two filtering algorithms are closely related. Could be nice to compare them in the control scheme eventually.
\begin{enumerate}
\item
Particle Filter: importance sampling with resampling.
\item
Gaussian Sum Filter: as described by Barber.
\end{enumerate}
\pagebreak

\subsection{April}
In this section we combine the preceding work. I think I'll stick to linear test problems in the beginning and venture into nonlinear problems later. 
\begin{figure}[H] 
\centering
\includegraphics[scale=1.0]{./imgs/controller}
\label{fig_cont_u}
\end{figure}
Note that I make room for model uncertainty and controller uncertainty... Perhaps they can be lumped? Will consider later. Try Wood Berry column in Huang's paper - uses BNs... Will look for a nice univariate problem as well.
\begin{enumerate}
\item
No switching states + deterministic controller: basically emulate PdV's Ph.D.
\item
Investigate uncertain controller? 
\item
Incorporate switching state
\end{enumerate}

\subsection{May}

\subsection{June}

\subsection{July}

\subsection{August}
USA? Maybe not?


\end{document}